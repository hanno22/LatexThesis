Das Quant einer kollektiven Anregung eines Plasmas bezeichnet man als Plasmon. Ein Plasma hat unterschiedliche Anregungsmoden, die sich aus den Maxwellgleichungen und den Bewegungsgleichungen der geladenen Teilchen des Plasmas ergeben. Jede dieser Anregungsmoden hat spezifische Dispersionrelationen, die die räumliche Ausdehnung der Welle in Abhängigkeit der Anregungsfrequenz beziehungsweise der Anregungsenergie festlegen.

Die einfachste Form der Anregung ist ein sogenanntes Volumenplasmon. Hierbei handelt es sich um eine kollektive Elektronendichtewelle. Durch die Bewegung eines Elektrons vor dem positiven Ionen-Hintergrundes entsteht ein Netto-Elektrisches Feld, welches als rücktreibene Kraft auf das Elektron wirkt. Durch den rein longitudinalen Charakter dieser rücktreibenen Kraft ist auch die Elektronendichtewelle rein longitudinal, daher ist es nicht möglich, ein Volumenplasmon mit Hilfe von elektromagnetischer Strahlung anzuregen, da Elektromagnetische Wellen im Vakuum rein transversal sind. Das heißt das sowohl das Elektrische, als auch das Magnetische Feld immer senkrecht auf der Ausbreitungsrichtung der Welle steht.

Betrachtet man nun statt einem unendlich ausgedehntem Plasma ein Plasma, das in einer Ebene an ein Dieelektrikum angrenzt, ergeben sich für die Grenzschicht weitere Anregunsmoden. Eine davon ist das sogenannte Surface-Plasmon-Polariton (SPP). Hierbei handelt es sich um eine Anregung, die in ihrer räumlichen Ausdehnung eng an die Grenzschicht zwischen Plasma und Dielektrikum gebunden ist. Die elektrische und die magnetische Feldstärke der Anregung fällt senkrecht zur Grenzschichtebene exponentiell ab. Man spricht von einem evaneszenten Feld. Das magnetische Feld dieser Mode ist rein Transversal, das elektrische Feld weist sowohl transversale als auch longitudinale Komponenten auf. Daher ist eine Kopplung zwischen SPP und Elektromagnetischer Strahlung also Photonen möglich. Aus den Maxwell-Gleichungen und den Randbedingungen an der Grenzschicht lässt sich die folgende Dispersionsrelation herleiten: $k(\omega) = ...$ Als Parameter wählen wir für das Plasma die Dielektrische Funktion von Gold und für das Dieelektrikum die Dieletrische Funktion von dem Glas des Substrates. Vergleicht man die so theoretisch berechnete Dispersion des SPP mit der Dispersion einer Elektromagnetischen Welle im Gold (ngold = ...), fällt auf, dass es zwischen den beiden Kurven keine Schnittpunkte gibt. Bei gleicher Anregungsfrequenz bzw. Energie haben die beiden Wellen unterschiedliche Räumlichefrequenzen. Dieser Missmatch in den Disipersionrelationen verhindert eine direkte Anregung, da zwischen Elektromagnetischer Welle und SPP beim ausbreiten auf der Probe eine zunehmende Phasendifferenz auftreten würde. Diese Problem kann durch unterschiedliche Mechanismen umgangen werden. 



Oberflächenplasmonen (Surface-Plasmon-Polaritonen) Sind kollektive Schwingungen von Elektronen und elektromagnetischen Feldern, die an der Grenzfläche zwischen einem Plasma und einem dielektrischen Material auftreten. Die Schwankungen der Elektronenladungsträgerdichte bilden nach den Maxwellgleichungen elektrische und Magnetische Felder aus. Die Amplitude der Elektrischen Feldstärke nimmt in beide Richtung mit zunehmender Entfernung von der Grenzflächen Exponentiell ab. Die Plasmonen breiten sich entlang der Grenzschicht aus. Oberflächenplasmonen haben im Gegensatz zu Volumenplasmonen sowohl Longitudinale, als auch Transversale Komponenten. Da elektromagnetische Wellen im Vakuum auschließlich eine transversale Elektrische Feldkomponente besitzten, können Photonen mit Oberflächenplasmonen wechselwirken, nicht jedoch mit Volumenplasmonen.